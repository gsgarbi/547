\def\emeas{\calE-measurable}
\def\BR{\borel{(\R)}}
\def\R{\bbR}
\def\ep{\calE}
This proof will use the following: \\
Prop. 2.3 from book with $B_0$ as the collection of sets of the form $(q, \infty)$. We know, from the previous question, that $B_0$ generates $\BR$.

$ (E, \cal{E}), (\R, \BR) $

$f: E \rightarrow \R$

$E = \bigcup_n{C_n}$

$C_i \cap C_j = \emptyset \ \forall i, j \geq 1 \land i \neq j $

\textbf{$ f \text{ is constant on each member of the partition} \implies f \text{ is } \calE \text{-measurable}  :$}

Assumption:\\
$ \forall i \in 1, 2, ...:$ \\
$f(x) = k_i \in \R \ \forall x \in C_i$


Now, for each set B of the form $(q, \infty)$:
$
\finv{B}
= 
\{
x \in E : f(x) \in B
\}
=
\{
x \in E : f(x) > q
\}
= \\
\bigcup C_{j}: k_j > q \ \forall j \in \bbN
$

I can write $\finv{B}$ in the way above because C is a partition of E.

Furthermore, since \cal{E} is the collection of countable unions of elements taken from C (with $\emptyset$ of course), we have that:
$\finv{B} \in \cal{E} $
$\forall B \in B_0$
\\

\textbf{$f \text{ is } \calE \text{-measurable} \implies f \text{ is constant on each member of the partition :}$}

Will prove the counter positive: \\
If $f$ is \textbf{NOT} constant on at least one member of the partition, then $f$ is \textbf{NOT} $\calE-measurable$.

Assume f is not constant on at least one member $C_d$ in C. This means:
$ \exists x_1, x_2 \in C_k: f(x_2) < f(x_1) $.

Then take $B = (q, \infty)$ with $f(x_2) < q < f(x_1)$.
\footnote{
This rational q exists because $f(x_1)$ and $f(x_2)$ are reals and Q is dense in $\R$, but I won't say q can be written as the average between those two values as said average may not be rational.
} 

Using the fact that C is a partition of E, we can write $\finv{B} = S_1 \cup S_2 $, where:

$S_1 = \bigcup C_{j}: k_j > q \ \forall j \in \bbN \setminus \{d\}$ and
$S_2 = \{ x \in C_d: f(x) \in B \} $ with $S_1 \cap S_2 = \emptyset $

$S_1$ is defined in a similar fashion as the first case and uses the fact that f takes constant values on partition elements, \textit{with the exception of $C_d$.}

However, $S_2$ is a \textbf{proper} subset of $C_d$ as its elements are taken from $C_d$ while $S_2$ can't be the whole $C_d$ since $x_2 \in C_d$, but $x_2 \not \in S_2$.

Since C is a partition of E and $S_2$ is a \textbf{proper} subset of $C_d$ , $\finv{B}$ will only be a member of $\cal{E}$, i.e a countable union of elements taken from C, if $S_2 = \emptyset$, but $x_1, x_2 \in S_2$, so $\finv{B} \not \in \cal{E}$.
\qed

