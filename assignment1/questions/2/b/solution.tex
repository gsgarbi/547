(2b)

$ \mu_n(A) = P(A), A \in \calC_n$

\textbf{Show that $\mu_n$ is a measure:}

Since P is a probability measure, $\mu_n: \calC_n \rightarrow [0, 1] \subset \R_{+} $

$\mu_n(\emptyset) = P(\emptyset) = 0 $ since P is a measure.

Let $\{A_n\} \subset C_n$ be a disjoint collection. Using that P is a measure: \\
$ \mu(\bigcup_n{A_n}) = 
\P(\bigcup_n{A_n}) = 
\sum_n{P(A_n)} =
\sum_n{\mu(A_n)}  $

Hence, $\mu_n$ is a measure.

\textbf{Not necessarily a probability measure:}

There's no guaranteed that $\mu_n(C_n) = 1$ so far. For instance, let $C = \{C_1, C_2\}, n=1, \mu_1(C_1) = 0.6, \mu_1(C_2) = 0.4$
Since $\mu_1(C_1) \neq 1$, $\mu_1$ is not a probability measure on $(C_1, \calC_1)$.

For $\mu_n$ to be a a probability measure on $(C_n, \calC_n)$, \textbf{we need} $\mu_n(C_n) = 1$, i.e, $ P(C_n) = 1 $.




